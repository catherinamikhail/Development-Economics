% Options for packages loaded elsewhere
\PassOptionsToPackage{unicode}{hyperref}
\PassOptionsToPackage{hyphens}{url}
%
\documentclass[
]{article}
\usepackage{amsmath,amssymb}
\usepackage{iftex}
\ifPDFTeX
  \usepackage[T1]{fontenc}
  \usepackage[utf8]{inputenc}
  \usepackage{textcomp} % provide euro and other symbols
\else % if luatex or xetex
  \usepackage{unicode-math} % this also loads fontspec
  \defaultfontfeatures{Scale=MatchLowercase}
  \defaultfontfeatures[\rmfamily]{Ligatures=TeX,Scale=1}
\fi
\usepackage{lmodern}
\ifPDFTeX\else
  % xetex/luatex font selection
\fi
% Use upquote if available, for straight quotes in verbatim environments
\IfFileExists{upquote.sty}{\usepackage{upquote}}{}
\IfFileExists{microtype.sty}{% use microtype if available
  \usepackage[]{microtype}
  \UseMicrotypeSet[protrusion]{basicmath} % disable protrusion for tt fonts
}{}
\makeatletter
\@ifundefined{KOMAClassName}{% if non-KOMA class
  \IfFileExists{parskip.sty}{%
    \usepackage{parskip}
  }{% else
    \setlength{\parindent}{0pt}
    \setlength{\parskip}{6pt plus 2pt minus 1pt}}
}{% if KOMA class
  \KOMAoptions{parskip=half}}
\makeatother
\usepackage{xcolor}
\usepackage[margin=1in]{geometry}
\usepackage{graphicx}
\makeatletter
\newsavebox\pandoc@box
\newcommand*\pandocbounded[1]{% scales image to fit in text height/width
  \sbox\pandoc@box{#1}%
  \Gscale@div\@tempa{\textheight}{\dimexpr\ht\pandoc@box+\dp\pandoc@box\relax}%
  \Gscale@div\@tempb{\linewidth}{\wd\pandoc@box}%
  \ifdim\@tempb\p@<\@tempa\p@\let\@tempa\@tempb\fi% select the smaller of both
  \ifdim\@tempa\p@<\p@\scalebox{\@tempa}{\usebox\pandoc@box}%
  \else\usebox{\pandoc@box}%
  \fi%
}
% Set default figure placement to htbp
\def\fps@figure{htbp}
\makeatother
\setlength{\emergencystretch}{3em} % prevent overfull lines
\providecommand{\tightlist}{%
  \setlength{\itemsep}{0pt}\setlength{\parskip}{0pt}}
\setcounter{secnumdepth}{-\maxdimen} % remove section numbering
\usepackage{bookmark}
\IfFileExists{xurl.sty}{\usepackage{xurl}}{} % add URL line breaks if available
\urlstyle{same}
\hypersetup{
  pdftitle={Group Data Assignment Week 3},
  pdfauthor={Catherina Mikhail, 2847118 Sinem Göral, 2817932 Ela Köycü 2846357},
  hidelinks,
  pdfcreator={LaTeX via pandoc}}

\title{Group Data Assignment Week 3}
\author{Catherina Mikhail, 2847118 Sinem Göral, 2817932 Ela Köycü
2846357}
\date{2025-11-12}

\begin{document}
\maketitle

1.a. Using the data files on Tanzania, we have computed the headcount,
poverty gap and squared poverty gap on the national level.

\begin{itemize}
\item
  We used the formula H = q/N, where q is the amount of people
  considered poor and N the total population. For 2012 we discovered
  that around 28.17\% of the population is considered poor using the
  headcount method. For 2018 we discovered that around 26.39\% of the
  population is considered poor using the headcount method.
\item
  We used the formula 1/N * sum(q * (z / (z - y))), where z is the
  poverty line and y is consumption, to compute the poverty gap. For
  2012 we discovered that the average poor person is falling short of
  the poverty line by 6.70\%. For 2018 we discovered that the average
  poor person is falling short of the poverty line by 6.16\%.
\item
  We used the formula 1/N * sum(q * ((z / (z - y))\^{}2)) to compute the
  squared poverty gap. For 2012 we discovered that on average, the
  income shortfall of poor households relative to the poverty line is
  12.82\% when giving greater weight to the poorest. For 2018 this was
  11.83\%.
\end{itemize}

1.b. Using the data files on Tanzania, we have computed the headcount,
poverty gap and squared poverty gap on the area level.

\begin{itemize}
\item
  Looking at the headcount per area we have discovered for 2012 33.29\%
  in rural areas, 21.69\% in urban areas and 4.11\% in Dar es Salaam are
  considered poor. For 2018 31.35\% in rural areas, 19.22\% in urban
  areas and 7.98\% in Dar es Salaam is considered poor.
\item
  Computing the poverty gap per area we found out that for 2012 the
  average poor person is falling short of the poverty line by 7.84\% in
  rural areas, 5.51\% in urban areas and 0.83\% in Dar es Salaam. For
  2018 the average poor person is falling short of the poverty line by
  7.40\% in rural areas, 4.16\% in urban areas and 2.03\% in Dar es
  Salaam.
\item
  Computing the squared poverty gap per area we discovered that for 2012
  on average, the income shortfall of poor households relative to the
  poverty line is 15.11\% in rural areas, 10.09\% in urban areas and
  1.69\% in Dar es Salaam. For 2018 on average, the income shortfall of
  poor households relative to the poverty line is 14.14\% in rural
  areas, 8.22\% in urban areas and 3.84\% in Dar es Salaam.
\end{itemize}

1.c. Using the data files on Tanzania, we have computed the headcount,
poverty gap and squared poverty gap on the region level.

\begin{itemize}
\item
  Looking at the headcount per area we have discovered for 2012 34.60\%
  in Dodoma, 21.89\% in Arusha, 11.38\% in Kilimanjaro, 30.74\% in
  Tanga, 11.13\% in Morogoro, 16.01\% in Pwani, 4.11\% in Dar es Salaam,
  17.74\% in Lindi, 41.39\% in Mtwara and 53.65\% in Ruvumaare
  considered poor. For 2018 23.19\% in Dodoma, 24.65\% in Arusha,
  10.52\% in Kilimanjaro, 20.95\% in Tanga, 19.31\% in Morogoro, 27.88\%
  in Pwani, 7.98\% in Dar es Salaam, 38.01\% in Lindi, 29.14\% in Mtwara
  and 30.60\% in Ruvumaare considered poor.
\item
  Computing the poverty gap per area we found out that for 2012 the
  average poor person is falling short of the poverty line by 7.09\% in
  Dodoma, 6.02\% in Arusha, 1.81\% in Kilimanjaro, 6.78\% in Tanga,
  2.38\% in Morogoro, 3.53\% in Pwani, 0.82\% in Dar es Salaam, 2.19\%
  in Lindi, 11.83\% in Mtwara and 15.55\% in Ruvumaare considered poor.
  For 2018 the average poor person is falling short of the poverty line
  by 4.05\% in Dodoma, 5.75\% in Arusha, 1.66\% in Kilimanjaro, 5.06\%
  in Tanga, 3.68\% in Morogoro, 8.73\% in Pwani, 2.03\% in Dar es
  Salaam, 9.58\% in Lindi, 5.93\% in Mtwara and 6.69\% in Ruvumaare
  considered poor.
\item
  Computing the squared poverty gap per area we discovered that for 2012
  on average, the income shortfall of poor households relative to the
  poverty line is 14.65\% in Dodoma, 10.70\% in Arusha, 4.12\% in
  Kilimanjaro, 13.52\% in Tanga, 4.78\% in Morogoro, 6.92\% in Pwani,
  1.69\% in Dar es Salaam, 5.70\% in Lindi, 20.17\% in Mtwara and
  26.99\% in Ruvumaare. For 2018 on average, the income shortfall of
  poor households relative to the poverty line is 8.82\% in Dodoma,
  11.07\% in Arusha, 3.83\% in Kilimanjaro, 9.62\% in Tanga, 7.75\% in
  Morogoro, 14.81\% in Pwani, 3.84\% in Dar es Salaam, 18.01\% in Lindi,
  11.75\% in Mtwara and 13.22\% in Ruvumaare.
\end{itemize}

\end{document}
